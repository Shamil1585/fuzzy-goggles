\documentclass{article}
\usepackage[T2A]{fontenc}
\usepackage[cp1251]{inputenc}
\usepackage{amsmath}
\usepackage{amsmath}
\usepackage{amsymb}
\usepackage{amsfonts}
\usepackage{mathrsfs}
\usepackage[12pt]{extsizes}
\usepackage{fancyvrb}
\usepackage{indentfirst}
\usepackage[
left=2cm, right=2cm, top=2cm, bottom=2cm, headsep=0.2cm, footskip=0.6cm, bindingoffset=0cm
]{geometry}
\usepackage[english,russian]{babel}


\begin{document}
\section*{Вариант 30}

Обозначим через \( \pi = (\pi_n), \, n \in B \), стационарное распределение вероятностей состояний процесса \( \Xi \) (сети \( N \)), а \( \hat{\pi} = (\hat{\pi}_n) \) и \( \tilde{\pi} = (\tilde{\pi}_n) \) — стационарные распределения вероятностей состояний цепей \( \tilde{C} \) и \( \tilde{C} \) соответственно. В стационарном режиме процесса \( \Xi \) вероятности начала и окончания тактов в состоянии \( s^{(n)} \in X \) равны вероятности \( \pi_n \), а вероятности окончания реализаций цепей \( \tilde{C} \) и \( \tilde{C} \) в состоянии \( n \in B \) равны вероятностям \( \hat{\pi}_n \) и \( \tilde{\pi}_n \). Поэтому, для \( n \in B \),

\[
\hat{\pi}_n = \sum_{m=1}^{c_Y} \pi_m p_{m,n}^{(\varphi)},
\]

\[
\tilde{\pi}_n = \sum_{m=c_Y+1}^{c_X} \pi_m p_{m,n}^{(\varphi)}, \tag{1}
\]

\[
\pi_n = \sum_{m=1}^{c_Y} \pi_m p_{m,n}^{(\varphi)} + \sum_{m=c_Y+1}^{c_X} \pi_m p_{m,n}^{(\varphi)}.
\]

Из конечности множества \( X \) непосредственно следуют существование и единственность распределения \( \pi \). Таким образом, при заданном значении \( \varphi \) стационарное распределение \( \pi \) вероятностей состояний сети \( N \) существует, является единственным и удовлетворяет системе уравнений (1) с условием \( \sum_{n \in B} \pi_n = 1 \).

\end{document}